%%%%%%%%%%%%%%%%%%%%%%%%%%%%%%%%%%%%%%%%%%%%%%%%%%%%%%%%%%%%%%%%%%%%%%%%%%%%%%%%%%%%%%%%%%%%%%%%
%
% CSCI 1430 Written Question Template
%
% This is a LaTeX document. LaTeX is a markup language for producing documents.
% Your task is to answer the questions by filling out this document, then to
% compile this into a PDF document.
%
% TO COMPILE:
% > pdflatex thisfile.tex

% If you do not have LaTeX, your options are:
% - VSCode extension: https://marketplace.visualstudio.com/items?itemName=James-Yu.latex-workshop
% - Online Tool: https://www.overleaf.com/ - most LaTeX packages are pre-installed here (e.g., \usepackage{}).
% - Personal laptops (all common OS): http://www.latex-project.org/get/ 
%
% If you need help with LaTeX, please come to office hours.
% Or, there is plenty of help online:
% https://en.wikibooks.org/wiki/LaTeX
%
% Good luck!
% The CSCI 1430 staff
%
%%%%%%%%%%%%%%%%%%%%%%%%%%%%%%%%%%%%%%%%%%%%%%%%%%%%%%%%%%%%%%%%%%%%%%%%%%%%%%%%%%%%%%%%%%%%%%%%
%
% How to include two graphics on the same line:
% 
% \includegraphics[width=0.49\linewidth]{yourgraphic1.png}
% \includegraphics[width=0.49\linewidth]{yourgraphic2.png}
%
% How to include equations:
%
% \begin{equation}
% y = mx+c
% \end{equation}
% 
%%%%%%%%%%%%%%%%%%%%%%%%%%%%%%%%%%%%%%%%%%%%%%%%%%%%%%%%%%%%%%%%%%%%%%%%%%%%%%%%%%%%%%%%%%%%%%

\documentclass[11pt]{article}

\usepackage[english]{babel}
\usepackage[utf8]{inputenc}
\usepackage[colorlinks = true,
            linkcolor = blue,
            urlcolor  = blue]{hyperref}
\usepackage[a4paper,margin=1.5in]{geometry}
\usepackage{stackengine,graphicx}
\usepackage{fancyhdr}
\setlength{\headheight}{15pt}
\usepackage{microtype}
\usepackage{times}
\usepackage{booktabs}

% From https://ctan.org/pkg/matlab-prettifier
\usepackage[numbered,framed]{matlab-prettifier}

\frenchspacing
\setlength{\parindent}{0cm} % Default is 15pt.
\setlength{\parskip}{0.3cm plus1mm minus1mm}

\pagestyle{fancy}
\fancyhf{}
\lhead{Final Project Progress Report}
\rhead{CSCI 1430}
\rfoot{\thepage}

\date{}

\title{\vspace{-1cm}Final Project Progress Report}

\begin{document}
\maketitle
\vspace{-1cm}
\thispagestyle{fancy}
**Important**: In your report, please
1) Make it very clear who on the team contributed what, and 
2) Include the dates of at least *two* meetings you had with your mentor.
\textbf{Team name: \emph{KENN}}\\
\textbf{TA name: \emph{Autumn Tilley}}

\emph{Note:} when submitting this document to Gradescope, make sure to add all other team members to the submission. This can be done on the submission page after uploading.

\section*{Progress Report Instructions}

Before writing your progress report, you should have met with your TA and talked through your progress.

\subsection*{Meeting dates}

1) TA meeting date: 4/19 5:30pm

2) Team meeting date: 4/23

\subsection*{Team contributions}

Please describe in one paragraph (3--4 sentences) per team member what each of you contributed to the project so far.
\begin{description}
\item[Person 1] I worked with Person 2 on preprocessing the image data that we got from Kaggle. We first downloaded the dataset that contains different folders for real and fake, train and validation data. Next, we initialized an ImageDataGenerator using Tensorflow's Keras library to augment and rescale the data. 
\item[Person 2] Person 1 and I worked on preprocessing. We separated the data into training, validation, and testing data. As well, we augmented the data in order to lay the foundation for subsequent analyses
\item [Person 3] I worked with person 4 on building the skeleton of our model. We also ran the model for few epochs to check inital results. The accuracy and the decreasing rate of loss seemed reasonable. 
\item [Person 4] I worked with person 3 on writing the code for our model and running it. I ran it several times and edited the code to improve our accuracy for the initial model. The accuracy and the decreasing rate of loss seemed reasonable. 
\end{description}

\end{document}